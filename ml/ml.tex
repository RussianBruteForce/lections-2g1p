\documentclass[oneside,final,14pt]{extreport}
\usepackage[utf8]{inputenc}
\usepackage[english,russian]{babel}
\usepackage{cmap}
\usepackage{mathtext}
\usepackage{amsmath, amsfonts, amssymb} 
\sloppy
\begin{document}
\title{Математическая логика и Теория Автоматов}

\maketitle

\chapter*{Вводный}
Формальная теория (Исчисление):
\begin{enumerate}
	\item Формулы
	\item Аксиомы
	\item Правила синтаксического вывода
\end{enumerate}
Доказательством в формальной теории называется последовательность формул, каждая из которых или является аксиомой, или получается из некоторых предыдущих по одному из правил вывода. Последней стоит формула, которую нужно доказать (\textbf{теорема} или $\vdash\Phi$).
$\vdash\Phi$ --- синтаксический вывод (дословно: доказуемая формула). Семантика \[f\colon\Phi\to\{true, false\}\].

Формула называется \textbf{тождественно истинной}, если она истинна при любой интерпритации.
$\models\Phi$ --- тождественно истинная формула.

Непротиворечивость:\[If \vdash \Phi, then \vDash \Phi\]

Полнота:\[If \vDash \Phi, then \vdash \Phi\]

Основная задача математической логики --- разработка и анализ формальных теорий, в т. ч. проверка их на непротиворечивость и полноту.

\chapter*{I раздел}
\section*{Исчисление высказываний (ИВ)}
\begin{itemize}
	\item $A, B, \dots$ --- пропозиционные переменные
	\item $\neg$ --- отрицание ("не")
	\item $\vee$ --- конъюнкция ("и")
	\item $\wedge$ --- дизъюнкция ("или")
	\item $\to$--- импликация ("если \dots, то \dots")
	\item $(),;\vdash$ --- остальные
\end{itemize}
Формулы:
\begin{enumerate}
	\item Пропозиционные переменные тоже формулы
	\item Если $\Phi$ и $\Psi$ формулы, то:
		\[\neg\Phi\]
		\[(\Phi \vee \Psi)\]
		\[(\Phi \wedge \Psi)\]
		\[(\Phi \to \Psi)\]
	\item $\Gamma \vdash \Phi$ --- список формул через запятую
	\item $\Gamma \vdash$ --- список формул из $\Gamma$ противоречив
\end{enumerate}

Аксиомы --- секвенции вида $\Phi \vdash \Phi$.

Доказательство в исчислении высказываний (ИВ) можно определить в виде дерева, узлы которого соответствуют правилам вывода (1-12), а ребра --- секвенции; вверху листья дерева --- аксиомы, внизу --- корень дерева (то, что нужно доказать).
\subsection*{Правила вывода}
\begin{enumerate}
	\item \[
	\cfrac{\Gamma \vdash \Phi; \Gamma \vdash \Psi}{\Gamma \vdash \Phi \wedge \Psi}
	\]
	\item \[
	\cfrac{\Gamma \vdash \Phi \wedge \Psi}{\Gamma \vdash \Phi}
	\]
	\item \[
	\cfrac{\Gamma \vdash \Phi \wedge \Psi}{\Gamma \vdash \Phi}
	\]
	\item \[
	\cfrac{\Gamma \vdash \Phi}{\Gamma \vdash \Phi \vee \Psi}	
	\]
	\item \[
	\cfrac{\Gamma \vdash \Psi}{\Gamma \vdash \Phi \vee \Psi}	
	\]
	\item \[
	\cfrac{\Gamma, \Phi \vdash X; \Gamma. \Psi \vdash X; \Gamma \vdash \Phi \vee \Psi}{\Gamma \vdash X}
	\]
	\item \[
	\cfrac{\Gamma, \Phi \vdash \Psi}{\Gamma \vdash \Phi \to \Psi}
	\]
	\item \[
	\cfrac{\Gamma \vdash \Phi; \Gamma \vdash \Phi \to \Psi}{\Gamma \vdash \Psi}
	\]
	\item \[
	\cfrac{\Gamma, \neg \Phi\vdash}{\Gamma \vdash \Phi}
	\]
	\item \[
	\cfrac{\Gamma \vdash \Phi; \Gamma \vdash \neg \Phi}{\Gamma \vdash}
	\]
	\item \[
	\cfrac{\Gamma_1, \Phi, \Psi, \Gamma_2 \vdash X}{\Gamma_1, \Psi, \Phi, \Gamma_2 \vdash X}
	\]
	\item \[
	\cfrac{\Gamma \vdash X}{\Gamma, \Psi \vdash X}
	\]
	Допустимые правила вывода:
	\item Транзитивность \[
	\cfrac{\Gamma_1 \vdash \Phi; \Gamma_2, \Phi \vdash \Psi}{\Gamma_1 \Gamma_2 \vdash \Psi}
	\]
\end{enumerate}

Правило вывода называется \textbf{допустимым}, если его использование не приводит к увеличению сножества используемых секвенций. \textbf{Они служат удобным сокращением фрагментов дерева доказательства.}


\end{document}