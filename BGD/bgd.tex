\textsl{}\documentclass[oneside,final,14pt]{extreport}
\usepackage[utf8]{inputenc}
\usepackage[english,russian]{babel}
\usepackage{cmap}
\usepackage{mathtext}
\usepackage{amsmath, amsfonts, amssymb}

\usepackage{xcolor}
\usepackage{hyperref}
% Цвета для гиперссылок
\definecolor{linkcolor}{HTML}{799B03} % цвет ссылок
\definecolor{urlcolor}{HTML}{799B03} % цвет гиперссылок
\hypersetup{pdfstartview=FitH,  linkcolor=linkcolor,urlcolor=urlcolor, colorlinks=true}

\sloppy
\begin{document}
\title{Гражданская оборона}

\maketitle

\chapter*{Гражданская защита в системе национальной безопасности в РФ}
Документы:
\begin{itemize}
	\item \href{http://kremlin.ru/acts/news/19653}{Концепция общественной безопасности в Российской Федерации}
	\item \href{http://www.rg.ru/2007/05/26/chs-dok.html}{Постановление Правительства Российской Федерации от 21 мая 2007 г. N 304}
	\item Учебник БЖД
\end{itemize}

Общественная безопасность --- это состояние защищенности человека и гражданина, материальных и духовных ценностей общества от преступных и иных противоправных посягательств, социальных и межнациональных конфликтов, а также ЧС природного и тезногенного характеров.

\section*{Основные угрозы общественной безопасности и направления по их ликвидации}
\begin{enumerate}
	\item Терроризм --- идеология насилия и практика воздействия на общественное сознание, на принятие решений органов государственной власти всех уровней, связанная с устранением населения или иными формами противоправных насильственных действий.
	\item Экстремизм --- деятельность национальных, религиозных, этнических и иных структур, направленная на нарушение единства и территориальной целостности РФ, дестабилизацию внутри социума в РФ.
	
	Профилактика экстремизма --- пропаганда патриотизма (бонов воспитывать?)
	\item Криминогенные угрозы. Якобы возникают из-за различных зависимостей и беспризорничества. Мусора, видимо, тоже: беспризорные наркоманы.
	\item Коррупция --- использование должностными лицами своих полномочий и прав, а также авторитета и связей в целях личной выгоды, противоречащей законодательству и моральным устоям.
	\item Миграция. МИГРАЦИЯ БЛЯДЬ!!!! СУКА! УГРОЗА! АЛЕРТА!
\end{enumerate}
\subsection*{ЧС техногенного характера}
\begin{enumerate}
	\item Аварии на объектах инфраструктуры и обеспечения жизнедеятельности населения.
	
	В рашке степерь износа подобных объектов = 70\%.
	
	\item Радиационная опасность.
	
	В т. ч. солнечное излучение, радиационный фон Земли и пр.
	
	\item Пожарная опасность
	
	\item Химическая и биологическая опасность
\end{enumerate}

\subsection*{ЧС природного характера}
\begin{enumerate}
	\item Гидрологическая опасность
	\item Геологическая опасность
	
	Основные направления деятельности для нейтрализации ЧС природного характера:
	\begin{itemize}
		\item Совершенствование системы РСЧС
		\item Мероприятия по ГО
		\item Исследование передового опыта иностранных государств
		\item Участие под эгидой международных организаций в миротворческих акциях
		
		Например вывозить из соседней страны заводы и трупы своих солдат на белых КамАЗах.
		
		\item Реконструкция пожарных депо, подготовка профессиональных пожарных
	\end{itemize}
\end{enumerate}

\section*{ЧС как фактор общественной безопасности}
\subsection*{Общая характеристика источников ЧС}
\textbf{ЧС} --- обстановка, сложившаяся на объекте определенной территории или акватории в результате возникновения источника ЧС, при котором нарушается нормальные условия деятельности и жизни людей, создается угроза жизни, хдоровью, есть пострадавшие, нанесен ущерб имуществу, объектам экономики.

\textbf{Источник ЧС} --- явление любой природы (в т. ч. и деятельность людей), в результате которого произошла или возможна ЧС.

Параметр $\to$ Фактор $\to$ Источник ЧС $\to$ ЧС

Негативные факторы:
\begin{itemize}
	\item Опасные
	
	Может привести или приводит к средней и легкой степени вреда
	
	\item Вредные
	
	Может привести к заболеванию в результате длительного воздействия.
	
	\item Поражающие
	
	К летальному исходу, тяжелая степень вреда или инвалидность, временная утрата жизнедеятельности. 
\end{itemize}
Негативные факторы по своей природе:
\begin{itemize}
	\item Физические (антропогенного и природного характеров)
	\item Химические (антропогенного и природного характеров)
	\item Биологические (антропогенного и природного характеров)
	\item Психофизиологические (нервные, психические и физические перегрузки)
\end{itemize}
\subsection*{Классификация ЧС по масштабу и тяжести последствий:}
\begin{table}[h]
	\begin{tabular}{|r|c|p{0.35\textwidth}|p{0.17\textwidth}|c|}
		\hline \textnumero & характер & территория & пострадавшие & ущерб \\
		\hline 1 & локальные & объект экономики & $\le10$ & $\le100\ 000$ \\
		\hline 2 & муниципальные &одно поселение или внутренняя территория города ФЗ & $\le50$ & $\le5$млн. \\
		\hline 3 & межмуниципальные & два и более поселений или территорий города ФЗ & $\le50$ & $\le5$ млн. \\
		\hline 4 & региональные & 1 субъект (из 85) РФ & $>50\ \le500$ & $5-500$ млн. \\
		\hline 5 & межрегиональные & 2 и более субъектов РФ & $>50\ \le500$ & $5-500$ млн. \\
		\hline 6 & федеральные & -- & $>500$ & $>500$ млн. \\
		\hline 7 & транс-граничные & -- & -- & -- \\
		\hline
	\end{tabular}
\end{table}


\chapter*{1.1 Система безопасности РФ}
\section*{Основные понятия и определения}

Безопасность --- состояние защищенности жизненно важных интересов личности, общества, государста от внутренних и внешних угроз/опасностей.

Безопасность --- реализованная деятельность по выявлению, идентификации и управлению уровнем воздействия на объект факторов источника опасности.

Безопасность --- состояние объекта, при котором риск нанесения ему вреда не превышает преемлимый уровень.

Приемлимый Rиск $R=1\cdot 10^{-6}$ человек в год

Жизненно важные интересы для личности:
\begin{itemize}
	\item реализация конституционных прав и возможностей
	\item качество жизни
	\item дузовное, интеллектуальное и физическое развитие
	\item здоровье
\end{itemize}

Жизненно важные интересы общества:
\begin{itemize}
	\item духовные и материальные ценности, их созранность
	\item демократия (лол)
	\item построение социального и правового государства (аахха, мы все в опасносте)
	\item достижение согласия
\end{itemize}

Жизненно важные интересы для государства
\begin{itemize}
	\item суверенитет
	\item международная безопасность
\end{itemize}

Принципы обеспечения безопасности:
\begin{itemize}
	\item Соблюдение и защита прав и свобод
	\item Законность
	\item Системпость и комплексность
	\item Приоритет и предусмотрение
	\item Взаимодействие органов власти с общественными организациями и международными системами безопасности
\end{itemize}

\section*{Огранизация и структура безопасности в РФ}
Субъект безопасности --- \textbf{государство}

Безопасность делится по:
\begin{itemize}
	\item объектам
	\item по характеру источника опасности (природная, технологическая, экологическая)
	\item по основному поражающему фактору (химическая, пожарная)
	\item по сфере деятельности государства (военная, экономическая)
\end{itemize}

Систему безопасности РФ образуют:
\begin{itemize}
	\item органы законодательной, исполнительной и судебной власти
	\item государственные, общественные и другие организации
	\item силы и средства обеспечения безопасности
	\item законодательство, регламентирующее отношения в сфере безопасности
\end{itemize}

\textbf{Система безопасности выполняет 11 задач!}

В релизации политики бещопасности принимают участие:\begin{itemize}
	\item президент
	\item федеральное собрание
	\item правительство
	\item федеральные органы исполнительнйо власти
	\item органы исполнительнйо власти субъектов \textbf{(их в РФ 85)}
	\item органы местного самоуправления
	\item первичные организации
\end{itemize}

Совет безопасности РФ состоит из постоянных членов (президент, представитель верховной и нижней палаты, силовые министры) и простых (губернаторы). \textbf{Министр МЧС -- не постоянный член.}

\section*{Стратегия, утвержденная указом президента РФ от 12 мая 2009 г. \#537}
Стратегия включает:
\begin{enumerate}
	\item Общие положения
	\item Соременный мир и роися
	\item Национальные интересы \textbf{(всего 9)}:
	\begin{enumerate}
		\item Ницаональная оборона
		\item Государственная и общественная безопасность
		\item Повышение качества жизни
		\item Экономический рост
		\item Развитие науки, технологий, образования
		\item Развитие здравоохранения
		\item Развитие культуры
		\item Экология живых систем
		\item Стратегия стабильности государства (ихихи).
	\end{enumerate}
	и стратегия национального приоритета:
	\begin{enumerate}
		\item Построение гражданского общества
		\item Обеспечение гос. бещопасности
		\item Превращение роиси в мировую державу.
	\end{enumerate}
	\item Обеспечение национальнйо безопасности
	\item Организационные, правовые и номартивные основы обеспечения стратегии
	\item Основные характеристи состояния национальнйо безопасности:
	\begin{enumerate}
		\item Децильный коэфициент (порог приблизительно 8\%) отношени 10\% самых обеспеченных к 10\% наименее обсеченных среди населения
		\item Уровень безработицы (порог 7\%)
		\item Инфляция (порог 20\%)
		\item Уровень обеспеченности ресурсами здравоохранения, науки, образования, культуры в процентном соотношении к ВВП. Порог:
		\begin{itemize}
			\item образование --- 1.5\%
			\item здравоохранение --- 1\%
			\item культура --- 0.5\%
			\item наука --- 2\%
		\end{itemize}
		\item Уровень ежегодного обновления военной и спец. тезники
		\item уровень обеспечения военными и спец. кадрами
	\end{enumerate}
\end{enumerate}

\chapter*{1.2 ПРАКТИКА. Угроза общественной безопасности в РФ}

\section*{Оснвоные угрозы общественной безопасности в РФ}

\section*{Черезвычайные ситуации --- фактор общественной бещопасности РФ}

\chapter*{1.3 РСЧС}


\chapter*{1.4 ПРАКТИКА}


\chapter*{1.5 Лекция}


\end{document}