\documentclass[oneside,final,14pt]{extreport}
\usepackage[utf8]{inputenc}
\usepackage[english,russian]{babel}
\usepackage{cmap}
\usepackage{mathtext}
\usepackage{amsmath, amsfonts, amssymb} 
\sloppy
\begin{document}
\title{ВОПРОСЫ \\
	для подготовки студентов к тестированию по дисциплине "Безопасность жизнедеятельности"}
\maketitle

\begin{enumerate}
	\item Количество признаков, позволяющих отнести событие к ЧС?
	\item В каком ГОСТе дается определение ЧС?
	\item  Что такое "источник ЧС"?
	\item  Что такое "безопасность"?
	\item  Чем характеризуется опасность?
	\item  Здоровье человека определяется
	\item  Объекты безопасности (Закон "О безопасности")?
	\item  Безопасность в ЧС - это...
	\item  Опасность в ЧС - это...
	\item  Поражающий фактор источника ЧС, чем он характеризуется?
	\item  Что обеспечивает безопасность для объектов безопасности?
	\item  Сколько принципов обеспечения безопасности (Закон "О безопасности")?
	\item  В вопросах обеспечения безопасности приоритетное положение дается
	\item  Является ли принципом безопасности взаимная ответственность личности, общества и государства в вопросах обеспечения безопасности?
	\item  Какие органы власти образуют систему безопасности в РФ?
	\item  Сколько основных функций выполняет система безопасности?
	\item  Какой орган координирует работу системы безопасности?
	\item  Относятся ли органы охраны природы к силам обеспечения безопасности в РФ?
	\item  Чем достигается безопасность в РФ?
	\item Кто возглавляет Совет безопасности РФ?
	\item  Когда принят Закон РСФСР "О безопасности"?
	\item  Когда принят Федеральный закон РФ "О защите населения и территории от ЧС природного и техногенного характера"?
	\item  Сколько основных мер защиты населения и территорий от ЧС определяет закон 40 защите населения и территорий от ЧС природного и техногенного характера"?
	\item  Каких органов государственной власти определяет полномочия Закон "О
	защите населения и территорий от ЧС природного и техногенного характера"?
	\item  Основные обязанности организаций в области защиты от ЧС включают несколько положений?
	\item  Перечень прав граждан в области защиты от ЧС содержит пунктов...
	\item  Перечень обязанностей граждан в области защиты от ЧС содержит положений,
	\item  Каким органом определяется порядок подготовки населения в области защиты от ЧС?
	\item  Когда принят Федеральный закон РФ" О гражданской обороне?
	\item 30 Принципы организации и ведения ГО (сколько)?
	\item  По какому принципу организуется ГО на территории РФ?
	\item  Сколько основных положений содержат обязанности граждан в области ГО?
	\item  Что создают организации, имеющие потенциально опасные производственные объекты для ведения ГО?
	\item  С какого момента начинается ведение ГО на территории РФ?
	\item  Кто по должности в организации является начальником ГО?
	\item  Номер Постановления правительства РФ "О порядке подготовки населения РФ в области защиты от ЧС"?
	\item  Когда принят Федеральный закон РФ "О промышленной безопасности опасных производственных объектов"?
	\item  Сколько: положений включают основы обеспечения промышленной безопасности опасных производственных объектов?
	\item  Какой документ издает МЧС для организации подготовки населения РФ в области защиты от ЧС?
	\item Субъект РФ принимает ли законы в области защиты населения от ЧС?
	\item  Из каких подсистем состоит РСЧС?
	\item  Сколько уровней имеет РСЧС?
	\item  Какой орган управления РСЧС является координирующим на федеральном уровне?
	\item  Какой орган управления РСЧС является координирующим на региональном уровне РСЧС?
	\item  Какой орган управления РСЧС является координирующим на территориальном уровне?
	\item  Какой орган управления РСЧС является постоянно действующим на федеральном уровне?
	\item  Какой орган управления РСЧС является постоянно действующим на региональном уровне?
	\item  Сколько региональных центров ГОЧС имеет Российская Федерация?
	\item  Количество субъектов РФ где созданы территориальные подсистемы РСЧС?
	\item Сколько режимов функционирования устанавливается для РСЧС?
	\item  Какие силы и средства имеет РСЧС?
	\item  К каким силам и средствам относятся формирования Всероссийской службы катастроф?
	\item  Во сколько эшелонов могут применяться силы РСЧС при ликвидации ЧС?
	\item  К какому региону РФ относится Омская область?
	\item  К каким органам управления РСЧС относятся пункты управления и центры управления в кризисных ситуациях?
	\item  Когда образован РСЧС и номер Постановления правительства РФ?
	\item  Какое должностное лицо исполнительной власти может быть председателем КЧС?
	\item  К каким силам РСЧС относятся формирования Государственного комитета санитарно-эпидемиологического надзора РФ?
	\item  Имеются ли силы постоянной готовности РСЧС?
	\item Имеются ли силы непостоянной готовности РСЧС?
	\item  Где создаются ТП РСЧС?
	\item  Сколько уровней имеет ТП РСЧС?
	\item  Какие органы управления имеет ТП РСЧС?
	\item  Где образуются звенья ТП РСЧС?
	\item  Какой орган управления ТП РСЧС относится к координирующему?
	\item  Какой орган управления ТП РСЧС относится к постоянно действующему?
	\item  Какой орган управления ТП РСЧС относится к повседневному?
	\item  Состав сил и средств ТП РСЧС.
	\item  Создается ли функциональная подсистема РСЧС в субъекте РФ?
	\item Какие организации образуют функциональную подсистему РСЧС?
	\item  В состав каких сил РСЧС входит "Госсанэпиднадзор"?
	\item  Сколько режимов функционирования устанавливается для ТП РСЧС?
	\item  К каким силам относятся ГУ ГО?
	\item  Кто является председателем KЧC в ОмГУ?
	\item  Какой орган управления звена ТП РСЧС ОмГУ является координирующим?
	\item  Какой орган управления звена ТП РСЧС ОмГУ является постоянно действующим?
	\item  Какой орган управления звена ТП РСЧС ОмГУ является повседневным?
	\item  Когда принят закон Омской области "О защите населения и территории Омской области от ЧС природного и техногенного характера"?
	\item  Кто является председателем КЧС ТП РСЧС Омской области?
	\item Каким органом власти определяется порядок подготовки населения в области защиты от ЧС?
\end{enumerate}

\end{document}
