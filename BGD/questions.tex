\documentclass[oneside,final,14pt]{extreport}
\usepackage[utf8]{inputenc}
\usepackage[english,russian]{babel}
\usepackage{cmap}
\usepackage{mathtext}
\usepackage{amsmath, amsfonts, amssymb} 
\sloppy
\begin{document}
\title{ВОПРОСЫ \\
	для подготовки студентов к тестированию по дисциплине "Безопасность жизнедеятельности"}
\maketitle

\section*{Херня с сайта bgd.omsu.ru}
\begin{enumerate}
	\item Количество признаков, позволяющих отнести событие к ЧС?
	\item В каком ГОСТе дается определение ЧС?
	\item  Что такое "источник ЧС"?
	\item  Что такое "безопасность"?
	\item  Чем характеризуется опасность?
	\item  Здоровье человека определяется
	\item  Объекты безопасности (Закон "О безопасности")?
	\item  Безопасность в ЧС - это...
	\item  Опасность в ЧС - это...
	\item  Поражающий фактор источника ЧС, чем он характеризуется?
	\item  Что обеспечивает безопасность для объектов безопасности?
	\item  Сколько принципов обеспечения безопасности (Закон "О безопасности")?
	\item  В вопросах обеспечения безопасности приоритетное положение дается
	\item  Является ли принципом безопасности взаимная ответственность личности, общества и государства в вопросах обеспечения безопасности?
	\item  Какие органы власти образуют систему безопасности в РФ?
	\item  Сколько основных функций выполняет система безопасности?
	\item  Какой орган координирует работу системы безопасности?
	\item  Относятся ли органы охраны природы к силам обеспечения безопасности в РФ?
	\item  Чем достигается безопасность в РФ?
	\item Кто возглавляет Совет безопасности РФ?
	\item  Когда принят Закон РСФСР "О безопасности"?
	\item  Когда принят Федеральный закон РФ "О защите населения и территории от ЧС природного и техногенного характера"?
	\item  Сколько основных мер защиты населения и территорий от ЧС определяет закон 40 защите населения и территорий от ЧС природного и техногенного характера"?
	\item  Каких органов государственной власти определяет полномочия Закон "О
	защите населения и территорий от ЧС природного и техногенного характера"?
	\item  Основные обязанности организаций в области защиты от ЧС включают несколько положений?
	\item  Перечень прав граждан в области защиты от ЧС содержит пунктов...
	\item  Перечень обязанностей граждан в области защиты от ЧС содержит положений,
	\item  Каким органом определяется порядок подготовки населения в области защиты от ЧС?
	\item  Когда принят Федеральный закон РФ" О гражданской обороне?
	\item Принципы организации и ведения ГО (сколько)?
	\item  По какому принципу организуется ГО на территории РФ?
	\item  Сколько основных положений содержат обязанности граждан в области ГО?
	\item  Что создают организации, имеющие потенциально опасные производственные объекты для ведения ГО?
	\item  С какого момента начинается ведение ГО на территории РФ?
	\item  Кто по должности в организации является начальником ГО?
	\item  Номер Постановления правительства РФ "О порядке подготовки населения РФ в области защиты от ЧС"?
	\item  Когда принят Федеральный закон РФ "О промышленной безопасности опасных производственных объектов"?
	\item  Сколько: положений включают основы обеспечения промышленной безопасности опасных производственных объектов?
	\item  Какой документ издает МЧС для организации подготовки населения РФ в области защиты от ЧС?
	\item Субъект РФ принимает ли законы в области защиты населения от ЧС?
	\item  Из каких подсистем состоит РСЧС?
	
	\textbf{Территориальные} подсистемы РСЧС (ТП РСЧС) создаются
	в субъектах РФ для предупреждения и ликвидации ЧС в пределах
	их территории и состоят из звеньев, соответствующих администра-
	тивно-территориальному делению этих территорий (более подроб-
	но организация и функционирование территориальной подсистемы
	РСЧС будут рассмотрены ниже).
	
	\textbf{Функциональные} подсистемы РСЧС (ФП РСЧС) \textbf{(>30)} создаются
	федеральными органами исполнительной власти для организации
	работы по защите населения и территорий от ЧС в сфере их дея-
	тельности и порученных им отраслях экономики.
	
	\item  Сколько уровней имеет РСЧС?
	
	РСЧС имеет \textbf{пять} уровней.
	
	\item  Какой орган управления РСЧС является координирующим на федеральном уровне?
	
	Правительственная комиссия по
	предупреждению и ликвидации чрезвычайных ситуаций и обеспе-
	чению пожарной безопасности (Правительственная КЧС и ПБ) и
	комиссии по предупреждению и ликвидации ЧС и обеспечению
	пожарной безопасности (КЧС и ПБ) федеральных органов испол-
	нительной власти;
	
	\item  Какой орган управления РСЧС является координирующим на региональном уровне РСЧС?
	
	полномочный представитель Президен-
	та РФ в федеральном округе;
	
	\item  Какой орган управления РСЧС является координирующим на территориальном уровне?
	
	КЧС и ПБ органа местного самоуправления;
	
	\item  Какой орган управления РСЧС является постоянно действующим на федеральном уровне?
	
	Министерство Российской Феде-
	рации по делам гражданской обороны, чрезвычайным ситуациям и
	ликвидации последствий стихийных бедствий (МЧС РФ); подраз-
	деления федеральных органов исполнительной власти для решения
	задач в области защиты населения и территорий от ЧС и (или) ГО;
	
	\item  Какой орган управления РСЧС является постоянно действующим на региональном уровне?
	
	территориальные органы МЧС
	России – главные управления МЧС РФ по субъекту РФ (ГУ МЧС по
	субъекту РФ); главные управления ГОЧС или МЧС субъекта РФ;
	
	\item  Сколько региональных центров ГОЧС имеет Российская Федерация?
	
	ХЗ(
	
	\item  Количество субъектов РФ где созданы территориальные подсистемы РСЧС?
	
	Хз(
	
	\item Сколько режимов функционирования устанавливается для РСЧС?
	
	\textbf{Три} режима: повсед, готовн и ЧС
	
	\item  Какие силы и средства имеет РСЧС?
	
	Наблюдения и контроля, ликвидации ЧС.
	
	\item  К каким силам и средствам относятся формирования Всероссийской службы катастроф?
	
	Силы и средства ликвидации ЧС.
	
	\item  Во сколько эшелонов могут применяться силы РСЧС при ликвидации ЧС?
	
	В \textbf{три} эшелона.
	
	\item  К какому региону РФ относится Омская область?
	
	Сибирский регион.
	
	\item  К каким органам управления РСЧС относятся пункты управления и центры управления в кризисных ситуациях?
	
	Органы повседневного управления.
	
	\item  Когда образован РСЧС и номер Постановления правительства РФ?
	
	No 794 от 30.12.2003 «О единой государственной системе предупреждения
	и ликвидации ЧС»;
	
	\item  Какое должностное лицо исполнительной власти может быть председателем КЧС?
	
	Комиссии
	возглавляются руководителями органов исполнительной власти и
	организаций или их заместителями.
	
	\item  К каким силам РСЧС относятся формирования Государственного комитета санитарно-эпидемиологического надзора РФ?
	
	Наблюдения и контроля.
	
	\item  Имеются ли силы постоянной готовности РСЧС?
	
	На каждом уровне РСЧС определены силы постоянной го-
	товности. Они предназначены для оперативного реагирования на
	ЧС. В состав этих сил входят аварийно-спасательные формирова-
	ния, укомплектованные с учётом обеспечения работы в автоном-
	ном режиме не менее трёх суток и находящиеся в состоянии посто-
	янной готовности.
	
	\item Имеются ли силы непостоянной готовности РСЧС?
	
	Нет.
	
	\item  Где создаются ТП РСЧС?
	
	Территориальные подсистемы РСЧС (ТП РСЧС) создаются
	
	\item  Сколько уровней имеет ТП РСЧС?
	
	РСЧС имеет \textbf{пять} уровней: федеральный, межрегиональный,
	региональный, муниципальный и объектовый.
	
	\item  Какие органы управления имеет ТП РСЧС?
	
	Постоянно действующие и повседневного управления
	
	\item  Где образуются звенья ТП РСЧС?
	
		в субъектах РФ для предупреждения и ликвидации ЧС в пределах
		их территории и состоят из звеньев, соответствующих администра-
		тивно-территориальному делению этих территорий.
	
	\item  Какой орган управления ТП РСЧС относится к координирующему?
	
	На федеральном, региональном, муниципальном и объекто-
	вом уровнях основным органом управления, ответственным за про-
	тиводействие ЧС, является КЧС соответствующего органа испол-
	нительной власти.
	
	\item  Какой орган управления ТП РСЧС относится к постоянно действующему?
	
	\begin{itemize}
	\item на федеральном уровне – Министерство Российской Феде-
	рации по делам гражданской обороны, чрезвычайным ситуациям и
	ликвидации последствий стихийных бедствий (МЧС РФ); подраз-
	деления федеральных органов исполнительной власти для решения
	задач в области защиты населения и территорий от ЧС и (или) ГО;
	\item на межрегиональном уровне – территориальные органы
	МЧС России – региональные центры по делам гражданской оборо-
	ны, чрезвычайным ситуациям и ликвидации последствий стихий-
	ных бедствий (РЦ ГОЧС);
	\item на региональном уровне – территориальные органы МЧС
	России – главные управления МЧС РФ по субъекту РФ (ГУ МЧС по
	субъекту РФ); главные управления ГОЧС или МЧС субъекта РФ;
	\item на муниципальном уровне – органы, специально уполномо-
	ченные на решение задач в области защиты населения и террито-
	рий от ЧС и (или) гражданской обороны (ГО) в органах местного
	самоуправления;
	\item на объектовом уровне – структурные подразделения орга-
	низаций, уполномоченных на решения задач в области защиты на-
	селения и территорий от ЧС и (или) ГО.
	
	\end{itemize}
	
	\item  Какой орган управления ТП РСЧС относится к повседневному?
	\begin{itemize} 
	\item федеральный уровень – национальный центр управления в
	кризисных ситуациях (национальный ЦУКС), ЦУКСы (ситуацион-
	но-кризисные центры), информационные центры, дежурно-диспет-
	черские службы федеральных органов исполнительной власти и
	уполномоченных организаций, имеющих ФП РСЧС);
	\item межрегиональный уровень – ЦУКСы региональных центров;
	\item региональный уровень – ЦУКСы ГУ МЧС по субъектам РФ;
	информационные центры, дежурно-диспетчерские службы органов
	исполнительной власти субъектов РФ и территориальных органов
	федеральных органов исполнительной власти.
	\item муниципальный уровень – единые дежурно-диспетчерские
	службы (ЕДДС) муниципальных образований;
	\item объектовый уровень – дежурно-диспетчерские службы
	(ДДС) организаций.
	
	\end{itemize}
	
	
	\item  Состав сил и средств ТП РСЧС.
	
	Состав сил и средств РСЧС определяется Постановлением
	Правительства РФ.
	
	\item  Создается ли функциональная подсистема РСЧС в субъекте РФ?
	
	Нет. Функциональные подсистемы РСЧС (ФП РСЧС) создаются
	федеральными органами исполнительной власти для организации
	работы по защите населения и территорий от ЧС в сфере их дея-
	тельности и порученных им отраслях экономики. А вот территориальные --- да.
	
	\item Какие организации образуют функциональную подсистему РСЧС?
	
	Хз(
	
	\item  В состав каких сил РСЧС входит "Госсанэпиднадзор"?
	
	Наблюдения и контроля.
	
	\item  Сколько режимов функционирования устанавливается для ТП РСЧС?
	
	\textbf{Три} режима.
	
	\item  К каким силам относятся ГУ ГО?
	
	Ликвидации ЧС.
	
	\item  Кто является председателем KЧC в ОмГУ?
	
	Ковалев
	
	\item  Какой орган управления звена ТП РСЧС ОмГУ является координирующим?
	\item  Какой орган управления звена ТП РСЧС ОмГУ является постоянно действующим?
	\item  Какой орган управления звена ТП РСЧС ОмГУ является повседневным?
	\item  Когда принят закон Омской области "О защите населения и территории Омской области от ЧС природного и техногенного характера"?
	\item  Кто является председателем КЧС ТП РСЧС Омской области?
	
	Зам губернатор.
	
	\item Каким органом власти определяется порядок подготовки населения в области защиты от ЧС?
	
	Порядок подготовки населения в области защиты от ЧС оп-
	ределяется Правительством РФ. Подготовка населения к действиям
	в ЧС осуществляется в организациях, в том числе в образователь-
	ных учреждениях, а также по месту жительства.
	
\end{enumerate}

\section*{Тест 1. Ключи алаха. (брутфорс 40 вариантов по БЖД)}
\subsection*{Правильные ответы на вопросы}
\begin{itemize}
	\item Какой из перечисленных показателей характеризует состояние национальной безопасности?
	
	Уровень безработицы.
	
	\item Стратегическая цель обеспечения экологической безопасности
	
	Сохранение окружающей природной среды и обеспеченеи её защиты.
	
	\item Виды безопасности (по сферам жизнедеятельности)?
	
	Экономическая, военная, социальная.
	
	\item К федеральным органам исполнительнйо власти относятся
	
	Министерства, ведомства, комитеты в составе правительства рашки
	
	\item К органам гос власти субъекта относятся? (хуёво, хуйво к ним относятся)
	
	Администрация субъекта и местное самоуправление
	
	\item Кто возглавляет совет безопасности?
	
	Председатель СБ
	
	\item Источник опасности --- это явление любой природы, способное привести к
	
	К реализации опасности
	
	\item Кто осуществляет общее руководство гос органами обеспечения безопасности?
	
	Председатель совета безопасности рашки
	
	\item Что такое безопасность?
	
	Опасность не превышает величины приемлимого риска
	
	\item Децильный коэфициент?
	
	Соотношение доходов 10\% самых бедных к 10\% самых наворовавших
	 
	
	
\end{itemize}




\subsection*{НЕправильные ответы на вопросы}

\begin{itemize}
	\item Чем достигается обеспечение национальной безопасности в ЧС?
	
	НЕ проведением единой гос политики
	
	\item Какие органы власти образуют систему безопасности РФ?
	
	НЕ федеральные, региональные, местные,объектовые
	
	\item Стратегическая цель обеспечения безопасности в сфере культуры
	
	НЕ сохранение самобытных культур рашкостана
	
	\item Кто осуществляет надзор за законностью в деятельнсоти органов обеспечения безопасности?
	
	Аллах. НЕ председатель  совета федерации фед. собрания рашки.
	
	\item По какому признаку создаются межведомственные комиссии Совета Безопасности?
	
	НЕ по территориальному.
	
	\item Основные приоритеты в обеспечении национальной безопасности
	
	НЕ развитие науки, техники, здравоохранения, образования (и плевать, что так написано в учебнике!)
	
	\item Сколько определено приоритетов для обеспечения национальнйо безопасности?
	
	НЕ три!
	
	\item Главная задача федерального собрания в сфере безопасности?
	
	НЕ контролировать силы и ср-ва, обеспечивающие безопасность рашки.
	
	\item Какие из перечисленных федеральных законов относятся к конституционным?
	
	НЕ <О безопасности>
	
	
	
	
	
	
\end{itemize}





\end{document}
