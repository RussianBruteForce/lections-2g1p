\documentclass[oneside,final,14pt]{extreport}
\usepackage[utf8]{inputenc}
\usepackage[english,russian]{babel}
\usepackage{cmap}
\usepackage{mathtext}
\usepackage{amsmath, amsfonts, amssymb} 
\sloppy
\begin{document}
\title{Дифференциальные Уравнения}

\maketitle

\chapter*{Лекция 1}
\section{Основные определения и понятия}
Обыкновенным \textbf{дифференциальным уравнением} называют равенство, в которое входят назависимая переменная, неизвестная функция и её (производные или дифференциалы). Общщий вид: \[F(x,y,y',y'',\ldots y^{(n)})=0\] или \[y^{(n)}=f(x,y,y',y'',\ldots y^{(n)})\]

\textbf{Порядком} фиференциального уравнения называется порядок старшей производной, входящей в уравнение.

\textbf{Степенью} дифференциального уравнения называется показатель степени старшей производдной, входящей в уравнение. Дифференциальные уравнения степень выше 1 сначала решают алгебраическими методами относительно старшей производной, а затем рассматривают дифференциальное уравнение 1 степени относительно старшей степени производной.

\textbf{Решением} дифференциального уравнения называется функция, которая при подстановке в уравнение обращает его в равенство.

Решение дифференциального уравнения, записаное в неявной форме, называется \textbf{интеграллом дифференциального уравнения}. Решения (интегралы) дифференциального уравнения разделяют на:
\begin{itemize}
	\item общие;
	\item частные;
	\item особые.
\end{itemize}

Общее решение чаще всего содержит всё многообразие решений и, помимо назависимой переменной, всегда содержит столько производных постоянных, каков поряддок дифференциального уравнения.
Частное решение находится из общего решения при конкретных численных значениях производных постоянных, входящих в него. Для выделения частного решения из общего, к дифференциальному уравнению задают дополнительные условия в количестве, равном порядку уравнения. Дополнительные уровнения к ДУ n--го порядка, заданные в виде значения функции и её производных до (n-1)--го порядка вклчительно, при одном и том же значении аргумента, называтся \textbf{начальным условием} или \textbf{условием Коши}.
Задача нахождения частного решения ДУ, удовлетворяющщего начальным условиям, наывается \textbf{задачей Коши}.
Решеия дифференциальных уравнений, которые не содержатся в общем решении (но они есть) называются особыми.
Вместо термина "решить дифференциальное уравнение" часто употребляется термин "\textbf{проинтегрировать ДУ}". Если решение ДУ можно записать через неопределенные или определенные интегралы, то употребляется термин "\textbf{ДУ интегрируемо в квадратурах}".

\section{Дифференциальные уравнения I порядка}
Рассмотрим ДУ: \[y'=f(x.y)\] начальное условие к нему имеет вид: \[y(x_0)=y_0\] т. е. $y=y_0$ при $x=x_0$.

Обозначим область определения функции $f(x,y)$ на области $O_xy$ как область $D$. С геометрической точки зрения ДУ в области $D$ определяет "\textbf{поле направлений}": в каждой точке этой области можно построить элемент касательной к неизвестной кривой, тоесть к решению ДУ, проходящего через эту точку. Начальное условие задает конкретную точку $M_0$.

График решения ДУ изображенный линиями называют \textbf{интегральными кривыми}.

Общее решение ДУ $y'=f(x,y)$ имеет вид \[y=\phi (x,C)\], где $C$ --- производная постоянная.Она определяет семейство интегральных кривых, которые, проходя через точки поля, имеют элементы касательных своими касательными.

\textbf{Частное решение} --- это одна интегральная кривая, проходящая через точку заданну начальным условием.

В некоторых случаях общее решение ДУ определяет такое семейство интегральных кривых, котороые имеют огибающую линию, которая касается всех линий семейства и будет особым решением уравнения. Особые решения могут быть молько у некоторых функций.

\dots

Теорема существования и единства решений











\end{document}
